В современном мире цифровые технологии упрощают и улучшают всё больше аспектов жизни человека, уже невозможно представить наш быт без мессенджеров, видеозвонков, покупок на онлайн маркетплейсах и интерактивных карт, способных в считанные секунды проложить маршрут из одного конца города в другой. Все эти действия позволяют сделать смартфоны, лежащие в кармане практически у каждого жителя мегаполиса.

Однако, пожелав посетить научную конференцию Яндекса, посвящённую передовым технологиям, я был вынужден искать конференц-зал по редким указателям, а также воспользоваться помощью сотрудников, подсказывающих куда именно надо идти. С аналогичной проблемой сталкивается большинство людей, пытающихся найти конкретное место в незнакомом для них помещении, будь то информационная стойка аэропорта или учебный кабинет в одном из корпусов университета.
И если крупные аэропорты с переменным успехом пытаются решать эту проблему, то студентам остаётся надеяться только на себя и на непривычные аналоговые указатели в новых корпусах.

Своим проектом я хочу предложить решение этой проблемы. Целью работы является разработка мобильного приложения для платформы iOS, отображающего карту помещений и прилегающей к ним территории, позволяющего пользователю производить поиск по карте и строить маршруты до интересующих его мест. В качестве местности был выбран кампус политехнического университета.

В процессе разработки необходимо решить следующие задачи:
\begin{itemize}
  \item Провести анализ предметной области и определить, какие технологии для решения этой проблемы уже существуют на рынке, определить их преимущества и недостатки
  \item Разработать методику высоко детализированной картографии и с её помощью составить план помещения и прилегающей территории
  \item Разработать мобильное приложение для операционной системы iOS, которое будет отображать карту и предоставлять пользователю удобный интерфейс для поиска кабинетов и построения маршрута
  \item Разработать возможность "поделиться"\ маршрутом с помощью графических кодов
\end{itemize}

Потребность разработки карты не только для помещений, но прилегающих территорий вытекает из банальной удобности использования единого приложения. При желании построить маршрут из кабинета одного корпуса до кабинета другого, пользователю будет достаточно проложить этот маршрут в одном приложении. Кроме того, внутренняя планировка кампуса университета на сторонних картах не отличается высокой точностью и достоверностью.

Возможность поделиться маршрутом с помощью графического представления будет актуальна для организаторов мероприятий, которые могут распечатать его на приглашениях.
