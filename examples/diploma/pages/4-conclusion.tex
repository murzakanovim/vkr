В ходе работы был проведён анализ существующих на рынке решений, рассмотрены их положительные и отрицательные стороны. На основе положительных сторон были составлены требования к разработке. Была разработана методика картографии на основе формата IMDF. Формат был расширен, был разработан редактор для расширенного формата \refAppendix{appendix:editor}, проведена ортофотограмметрия \refAppendix{appendix:orto} и составлена детализированная карта кампуса СПбПУ.

В результате работы было разработано мобильное приложение для операционной системы iOS, удовлетворяющее всем стандартам качественной современной разработки, в том числе поддержка всех видов актуальных устройств, светлой и тёмной темы, интернационализации. Скриншоты приложения в приложение \ref{appendix:ios-result}.

Приложение позволяет пользователю просматривать карту используя, для этого привычные жесты управления, осуществлять поиск по аннотациям и строить маршруты, бесшовно переходящие с улицы в помещения.

После построения маршрута пользователь может создать красиво оформленный QR или AppClip код с заложенным маршрутом.


\section*{Планы на развитие}
  В дальнейшем планируется выпустить унифицированную версию приложения, с возможностью скачивать карты из интернета и после этого отображать их. Для неё будет необходимо разработать веб версию конструктора карт, чтобы каждый желающий смог построить план интересующего его заведения и открыть его в универсальном приложении. Кроме того, я планирую выпустить Android версию приложения.
